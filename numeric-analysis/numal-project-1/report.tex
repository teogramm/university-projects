%!TEX program = xelatex
\documentclass[12pt,a4paper]{article}
\usepackage{polyglossia}
\setmainlanguage[variant=mono]{greek}
\setotherlanguage{english}
\setmainfont[Mapping=tex-text]{GFS Elpis}
\usepackage{amsmath}
\usepackage{amsfonts}
\usepackage{amssymb}
\usepackage{amsthm}
\usepackage{tikz}
\usepackage{pgfplots}
\usepackage{listings}
\usepackage{float}
\pgfplotsset{compat=1.16}
\author{Γραμμένος Θεόδωρος}
\title{1η υποχρεωτική εργασία στο μάθημα της Αριθμητικής Ανάλυσης}
\date{ΑΕΜ: 3294}
\begin{document}
    \maketitle
    \section{Άσκηση 1}
    \begin{center}     
        \begin{tikzpicture}
            \begin{axis}[xmin=-2.5,xmax=2.5,ymin=-2,ymax=40,samples=150,scale only axis,grid=major,axis lines=middle,title={Γραφική παράσταση της f(x)},]
            \addplot[blue,domain=-2:2](x,{e^((sin(deg(x)))^3) + x^6 - 2*x^4 - x^3 - 1});
            \end{axis}
        \end{tikzpicture}
    \end{center}    
    
    Από τη γραφική παράσταση της f μπορούμε να δούμε ότι υπάρχουν 3 ρίζες: μία στο διάστημα $[-2,-1]$, μία στο διάστημα $[-\frac{1}{2},\frac{1}{2}]$ και μία 
    στο διάστημα $[1,2]$.

    Αρχικά θέλουμε να υπολογίσουμε τη ρίζα στο διάστημα $[-2,-1]$. Συνεπώς τρέχουμε τις συναρτήσεις bisection, newton και secant. Στην bisection δίνουμε ως 
    είσοδο το διάστημα $[-2,-1]$, στη newton $x_0 = -2$ και στη secant $x_0=-2 \text{ και } x_1=-1.5$. Αντίστοιχα για τη ρίζα στο $[1,2]$ δίνουμε ως είσοδο 
    στη newton $x_0 = 2$ και στη secant $x_0 = 2 \text{ και } x_1=1.5$. Για την ρίζα κοντά στο 0, δίνουμε  στη newton $x0 = 0.5$ και στη secant $x_0 = 0.5 
    \text{ και } x_1=0.25$. Σε όλες δηλώνουμε επιθυμητή ακρίβεια $10^{-5}$. Τα αποτελέσματα φαίνονται στον παρακάτω πίνακα:

    \begin{table}[hb]
        \centering
        \footnotesize
        \setlength\tabcolsep{1.5pt}
        \begin{tabular}{|l|c|c|c|c|c|c|l}
        \cline{1-7}
        Διάστημα  & \multicolumn{2}{c|}{$[-2,-1]$}                                     & \multicolumn{2}{c|}{$[-0.5,0.5]$}                                  & \multicolumn{2}{c|}{$[1,2]$}                                       &  \\ \cline{1-7}
                  & \multicolumn{1}{l|}{Αποτέλεσμα} & \multicolumn{1}{l|}{Επαναλήψεις} & \multicolumn{1}{l|}{Αποτέλεσμα} & \multicolumn{1}{l|}{Επαναλήψεις} & \multicolumn{1}{l|}{Αποτέλεσμα} & \multicolumn{1}{l|}{Επαναλήψεις} &  \\ \cline{1-7}
        bisection & -1.1976                         & 16                               & -                               & -                                & 1.5301                          & 16                               &  \\ \cline{1-7}
        newton    & -1.1976                         & 8                                & 6.9657e-05                      & 31                               & 1.5301                          & 6                                &  \\ \cline{1-7}
        secant    & -1.1976                         & 8                                & 7.7577e-05                      & 41                               & 1.5301                          & 5                                &  \\ \cline{1-7}
        \end{tabular}
        \caption{Αποτελέσματα όλων των μεθόδων}
        \label{tab:root-results}
    \end{table}

    Βλέπουμε ότι τη ρίζα στο $[-2,-1]$ την εντοπίζουν και οι 3 στην ίδια τιμή, με την μέθοδο της διχοτόμησης να κάνει τις περισσότερες επαναλήψεις (16), ενώ 
    η Newton-Raphson και η μέθοδος της τέμνουσας συγκλίνουν πολύ ταχύτερα, βρίσκοντας τη ρίζα με 8 επαναλήψεις. Το ίδιο ισχύει και για τη ρίζα 
    στο διάστημα $[1,2]$. Όμως στο διάστημα $[-\frac{1}{2},\frac{1}{2}]$ τα αποτελέσματα διαφέρουν αρκετά. Η μέθοδος της διχοτόμησης δεν μπορεί να βρει τη ρίζα, 
    αφού το πρόσημο της f δεν αλλάζει. Οι μέθοδοι Newton-Raphson και της τέμνουσας βρίσκουν 2 αριθμούς πολύ κοντά στο 0, αλλά κάνουν 
    πολύ περισσότερες επαναλήψεις σε σχέση με τις άλλες ρίζες. Επίσης η μέθοδος της τέμνουσας κάνει περισσότερες επαναλήψεις από την Newton-Raphson και 
    παρ'όλα αυτά βρίσκει μια ρίζα που απέχει περισσότερο από το 0(που είναι η πραγματική ρίζα) σε σχέση με την ρίζα που βρίσκει η Newton-Raphson.

    Παρατηρούμε ότι η μέθοδος Newton-Raphson βρίσκει τις ρίζες στα διαστήματα $[-1,-2]$ και $[1,2]$ πολύ γρηγορότερα από τη μέθοδο της διχοτόμησης, ενώ την άλλη 
    ρίζα που βρίσκεται στο 0 την προσεγγίζει, αλλά δεν καταφέρνει να την βρει. Το Θεώρημα 2.2 του βιβλίου λέει ότι η μέθοδος του Νεύτωνα συγκλίνει τετραγωνικά 
    όταν για μια ρίζα $x^*$ όταν $f'(x^*) \neq 0$ και το $x_0$ ανήκει σε ένα διάστημα με μέσο το $x^*$. Προσθέτει ότι αν $f''(x^*) \neq 0$ τότε η τάξη σύγκλισης 
    είναι ακριβώς 2. Γεωμετρικά, μπορούμε να πούμε ότι δεν θέλουμε η $f$ στο $x^*$ να παρουσιάζει τοπικό ακρότατο και σημείο καμπής. Βλέποντας ξανά τη γραφική 
    παράσταση της $f$ μπορούμε να δούμε ότι οι ρίζες στα $[-1,-2]$ και $[1,2]$ πράγματι δεν είναι ούτε τοπικά ακρότατα ούτε σημεία καμπής και συνεπώς η μέθοδος του 
    Νεύτωνα συγκλίνει τετραγωνικά για τις ρίζες σε αυτά τα διαστήματα. Το 0 όμως παρατηρούμε ότι είναι τοπικό ακρότατο, συνεπώς το θεώρημα δεν ικάνοποιείται και 
    δεν μπορούμε να ισχυριστούμε ότι η Newton-Raphson έχει τετραγωνική σύγκιση για αυτήν την ρίζα.

    \section{Άσκηση 2}
    Από τη γραφική παράσταση της δοθείσας $f$ μπορούμε να δούμε ότι η f έχει 5 ρίζες στα διαστήματα: $[-2,-1], [-1,0],[0,0.4],[0.4,0.7] \text{ και } [1,2]$.
    Για να βρούμε τη ρίζα στο $[-2,-1]$ χρησιμοποιούμε την τροποποιημένη Newton-Raphson με αρχικό σημείο το $-2$. Βρίσκουμε $r_0 = -1.3813$ μετά από 5 επαναλήψεις.
    Για να βρούμε τη ρίζα στο $[1,2]$ χρησιμοποιούμε τη μέθοδο της τέμνουσας με $x_{n+2} = 2, x_{n+1} = 1.5 \text{ και } x_n = 1$. Βρίσκουμε $r_1 = 17.61$ μετά 
    από 23 επαναλήψεις. Για να βρούμε τις ρίζες στα $[0,0.4]$ και $[0.4,0.7]$ μπορούμε να χρησιμοποιήσουμε την τροποποιημένη μέθοδο διχοτόμησης αφού βλέπουμε πως 
    η συνάρτηση αλλάζει πρόσημο εκατέρωθεν των ριζών, όπως φαίνεται στο παρακάτω γράφημα: 

    \begin{center}     
        \begin{tikzpicture}
            \begin{axis}[xmin=-1,xmax=1,ymin=-25,ymax=2,samples=150,scale only axis,grid=major,axis lines=middle,title={Γραφική παράσταση της f(x) στο [-1,1]},]
            \addplot[blue,domain=-1:1](x,{54*x^6 + 45*x^5 - 102*x^4 - 69*x^3 + 35*x^2 + 16*x - 4});
            \end{axis}
        \end{tikzpicture}
    \end{center}

    Επίσης, στο γράφημα μπορούμε να δούμε ότι η ρίζα στο $[-1,0]$ έχει άρτια πολλαπλότητα αφού το πρόσημο δεν αλλάζει εκατέρωθέν της. Συνεπώς, για να βρούμε τη 
    ρίζα θα εφαρμόσουμε την τροποποιημένη Newton-Raphson με $x_0 = -0.4$. Έτσι έχουμε τις υπόλοιπες ρίζες:

    \begin{table}[H]
        \centering
        \begin{tabular}{|l|l|l|}
        \hline
        Διάστημα    & Ρίζα    & Επαναλήψεις \\ \hline
        $[-1,0]$    & -0.6667 & 10          \\ \hline
        $[0,0.4]$   & 0.2052  & 25          \\ \hline
        $[0.4,0.7]$ & 0.5     & 22          \\ \hline
        \end{tabular}
        \caption{Οι υπόλοιπες ρίζες της $f$}
        \label{tab:f-roots}
    \end{table}

    Για να διαπιστώσουμε εάν η τροποποιημένη μέθοδος διχοτόμησης συγκλίνει με σταθερό αριθμό επαναλήψεων τρέχουμε τον κώδικα:
    \begin{lstlisting}
        for i = 1:10
            [root,reps] = modBisection(0.4,0.7,10^-5)
            repsArray(i) = reps
        end
    \end{lstlisting}
    και παίρνουμε:
    \begin{lstlisting}
        repsArray =

            12 18 22 15 15 22 17 18 21 24
    \end{lstlisting}
    Βλέποντας το repsArray μπορούμε να δούμε ότι η τροποποιημένη μέθοδος διχοτόμησης δεν συγκλίνει με σταθερό αριθμό επαναλήψεων. Αυτό 
    συμβαίνει αφού σε κάθε επανάληψη παίρνει ένα τυχαίο σημείο στο διάστημα που εξετάζει και συνεπώς υπάρχουν φορές που τα σημεία τα 
    οποία διαλέγει είναι πιο κοντά στη ρίζα, άρα έχουμε ταχύτερη σύγκλιση, ενώ άλλες φορές επιλέγει σημεία που κάνουν τη σύγκλιση πιο 
    αργή.

    \begin{table}[hb]
        \centering
        \setlength\tabcolsep{1.8pt}
        \begin{tabular}{|l|l|l|l|l|l|l|}
        \hline
        Διάστημα    & \multicolumn{2}{l|}{Διχοτόμηση} & \multicolumn{2}{l|}{Newton} & \multicolumn{2}{l|}{Tέμνουσα} \\ \hline
                    & Αρχική        & Αλλαγμένη       & Αρχική      & Αλλαγμένη     & Αρχική       & Αλλαγμένη      \\ \hline
        $[-2,1]$    & 16            & 26.3            & 8           & 5             & 22           & 173            \\ \hline
        $[-1,0]$    & -             & -               & 14          & 10            & 19           & 110            \\ \hline
        $[0,0.4]$   & 15            & 25              & 3           & 3             & 4            & 6              \\ \hline
        $[0.4,0.7]$ & 14            & 22.5            & 5           & 3             & 6            & 9              \\ \hline
        $[1,2]$     & 16            & 22.5            & 8           & 5             & 14           & 23             \\ \hline
        \end{tabular}
        \caption{Σύγκριση μεθόδων}
        \label{tab:method-comparison}
    \end{table}
    Για την τροποποιημένη μέθοδο διχοτόμησης χρησιμοποιήθηκε ο μέσος όρος από 4 επαναλήψεις. Σε όλες τις συναρτήσεις δόθηκαν τα ίδια ορίσματα 
    στην αρχική και την τροποποιημένη μέθοδο.

    \section{Άσκηση 3}
    Τα .m αρχεία που υλοποιούν τα ζητούμενα βρίσκονται στο φάκελο \\ scripts/Ex3.

    \section{Άσκηση 4}
    \subsection*{Zητούμενο 1}
        Θα αποδείξουμε ότι ο πίνακας G είναι στοχαστικός, δηλαδή το άθροισμα κάθε στήλης του είναι 1. Για ένα στοιχείο στη θέση (i,j) του πίνακα G ισχύει: 
        $G_{i,j} = \frac{q}{n} + \frac{A_{j,i}(1-q)}{n_j}$. Ουσιαστικά πρέπει να αποδείξουμε ότι: $\sum\limits_{k=1}^n G_{k,j} = 1$.
        \begin{proof}
            $\sum\limits_{k=1}^n G_{k,j} = \sum\limits_{k=1}^n \left(\frac{q}{n} + \frac{A_{j,k}(1-q)}{n_j}\right) = \sum\limits_{k=1}^n \frac{q}{n} + 
            \sum\limits_{k=1}^n \left(\frac{q}{n} + \frac{A_{j,k}(1-q)}{n_j}\right) = n\frac{q}{n} + \frac{1-q}{n_j} \sum\limits_{k=1}^n A_{j,k}$ Όμως 
            ισχύει ότι $n_j$ είναι το άθροισμα της j-οστής γραμμής του πίνακα Α και άρα μπορεί να γραφεί ως: $\sum\limits_{k=1}^n A_{j,k}$. Έτσι έχουμε: 
            $n\frac{q}{n} + \frac{1-q}{n_j} \sum\limits_{k=1}^n A_{j,k} = q + (1-q)\frac{\sum\limits_{k=1}^n A_{j,k}}{\sum\limits_{k=1}^n A_{j,k}} = 
            q + (1-q) = 1$.
        \end{proof}
    \subsection*{Zητούμενο 2}
        Χρησιμοποιώντας τη συνάρτηση powerMet   hod που υλοποιήθηκε με όρισμα τον πίνακα G μπορούμε να δούμε ότι πράγματι η μέγιστη ιδιοτιμή είναι 1. Για να ελέγξουμε 
        το ιδιοδιάνυσμα που δίνεται στην άσκηση, παίρνουμε το evector που δίνεται ως έξοδος από τη συνάρτηση και το κανονικοποιούμε ως προς την $L_1$ νόρμα του, αφού 
        θέλουμε το άθροισμα των στοιχείων του να είναι 1. Έτσι έχουμε:
        \begin{lstlisting}
        evector = evector/norm(evector,1)
            evector =
                0.0268
                0.0299
                ...
                0.1163
                0.1251
        \end{lstlisting}
        Το ιδιοδιάνυσμα είναι πράγματι αυτό που δίνεται στην άσκηση.

    \subsection*{Zητούμενο 3}
        Θα προσθέσουμε τις συνδέσεις $\left\{4,8\right\}, \left\{11,8\right\}, \left\{7,8\right\}, \left\{15,8\right\}$ και \\αφαιρούμε την $\left\{8,11\right\}$, 
        προκειμένου να βελτιώσουμε τη βαθμολογία της σελίδας 8. Υλοποιούμε αυτές τις αλλαγές στη συνάρτηση createImprovedMatrixG. Μετά τις αλλαγές η σελίδα 4 
        έχει βαθμό σημαντικότητας 0.1170 ενώ πριν είχε 0.0396.


    \subsection*{Ζητούμενο 4}
        Τρέχοντας τη συνάρτηση createImprovedMatrixG με όρισμα $q_0 = 0.02\\ \text{και } q_1 = 0.6$ παίρνουμε τα ιδιοδιανύσματα και αφού τα κανονικοποιήσουμε 
        παρατηρούμε ότι για πιθανότητα μεταπήδησης $q_0 = 0.02$ έχουμε $p_0 = 0.1332$ και για $q_1 = 0.6, p_1 = 0.0883$ για τη σελίδα 8, την οποία βελτιώσαμε. 
        Παίρνουμε μια άλλη σελίδα στην οποία δεν δείχνουν πολλές σελίδες, π.χ. τη σελίδα 1. Για την σελίδα 1 έχουμε συνοπτικά: $p_{orig} = 0.0265, p_{0} = 
        0.0159, p_1 = 0.0513$. Έτσι συμπεραίνουμε ότι η πιθανότητα μεταπήδησης δείχνει το πόσο πιθανό είναι ο χρήστης να πάει σε κάποια σελίδα, όχι επειδή 
        πάτησε κάποιο σύνδεσμο σε μία άλλη σελίδα, αλλά επειδή την ανακάλυψε με κάποιον άλλο τρόπο. Ίσως να του την είπε κάποιος φίλος του ή να την είδε 
        κάπου γραμμένη. Πάντως είναι σίγουρο πως κάποιος δεν μπαίνει σε μια σελίδα μόνο επειδή πάτησε ένα σύνδεσμο. Αυτό φαίνεται και στις πιθανότητες που 
        παρουσιάστηκαν παραπάνω. Για τη σελίδα 8, στην οποία δείχνουν πολλές σελίδες, όταν μειώνεται η πιθανότητα μεταπήδησης αυξάνεται η τάξη της σελίδας, ενώ 
        όταν αυξάνεται η πιθανότητα μεταπήδησης μειώνεται η τάξη της, αφου δεν παίζει τόσο ρόλο πόσες σελίδες δείχνουν σε αυτή, καθώς η πιθανότητα να επισκευτεί 
        κάποιος μια σελίδα είναι ήδη μεγάλη. Για τη σελίδα 1, στην οποία δείχνει μόνο μία άλλη σελίδα, όταν μειώνεται η πιθανότα μεταπήδησης μειώνεται η 
        τάξη της, ενώ όταν αυξάνεται η πιθανότητα αυξάνεται και η τάξη της. Αφού λίγες σελίδες δείχνουν στη σελίδα 1, η τάξη της επηρεάζεται περισσότερο από 
        την πιθανότητα να την επισκευτεί κάποιος για κάποιο άλλο λόγο και όχι επειδή ακολούθησε έναν σύνδεσμο.

    \subsection*{Zητούμενο 5}
        Αφού κάνουμε τις αλλαγές στον πίνακα και υπολογίσουμε τις τάξεις των σελίδων μετά τις αλλαγές, η τάξη της σελίδας 11 όντως βελτιώνεται από 0.1063 
        σε 0.1136, αν και η διαφορά είναι πολύ μικρή. Την ίδια στιγμή η τάξη της σελίδας 10 μειώνεται ελάχιστα από 0.1063 σε 0.1062. Άρα η στρατηγική αυτή 
        δουλεύει, παρ'όλα αυτά θα είχε καλύτερα αποτελέσματα αν η σελίδα 11 μπορούσε να πείσει και άλλες σελίδες να την προβάλλουν πιο ευνοϊκά.
    
    \subsection*{Ζητούμενο 6}
        Παρατηρούμε ότι αυξάνονται οι τάξεις όλων των σελίδων εκτός από τις 14 και 15.
\end{document}